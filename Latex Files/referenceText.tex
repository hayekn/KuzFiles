\documentclass[center]{hayektex}

\title{}
\subtitle{A subtitle}
\note{And a note}
\email{you@yourweb.com}
\author{Your name}
\newcommand{\mblue}[1]{\textcolor{blue}{#1}}
\MakeOuterQuote{"}

\begin{document}\renewcommand{\arraystretch}{2}
	\section{June 11, Literature Review}
	\\\textit{Binge Drinking}
	\\Though global definitions remain non-standardized, in the United States, binge drinking refers to an "excessive episodic consumption of alcohol," usually 5 or 4 drinks, for men and women, respectively \cite{stolle}, but see also the "4/5" rule \cite{fillmore}.
	\\\textit{Compulsivity}
	\\Compulsivity describes the "urge [in a subject] to perform an overt or covert behavior" despite a "lack of goal orientation." Thus, subjects may face "adverse consequences" as a result of compulsive behavior. Importantly, compulsivity is a key driver of alcohol- and drug-use disorders \cite{burchi}.
	\\\textit{Insula}
	\\In \cite{naqvi}, researchers observed that the "agranular insula... [may] play a part in pain modulation and the rewarding effects of some drugs of abuse." Furthermore, "functional imaging studies have revealed activation of the insula during drug urges." This is not so surprising, as the insula is deeply involved in "salience and reward networks" \cite{radhakrishnan}.
	\\Further imaging studies have related the "anterior insular cortex [to] impulse control in both healthy individuals and those addicted to drugs" \cite{belin, dambacher, ersche}. It has also been observed that the same neurons in the AIC which fire in response to the onset of drinking (in rats) also "had greater activity under compulsion-like conditions" \cite{hopf}.
	\newpage\margintext{especially with respect to water-fed control subjects. notes the varying timescales used by researchers: for an alcohol access period of 2 hours, 10 minute- up to 40 minute-periods have been considered}

%	The insula processes sensory information from the body \cite{homma}. Involved in "salience and reward networks" \cite{radhakrishnan}."Many functional imaging studies have revealed activation of the insula during drug urges"; "the insula is necessary for the explicit motivation to take drugs"; "The agranular insula also contains a high concentration of endogenous opioids and a high density of $\mu$-opioid receptors, which might play a part in pain modulation and the rewarding effects of some drugs of abuse" \cite{naqvi}.
%	
	\bibliographystyle{apalike}
	\bibliography{ref.bib}{}
	
\end{document}